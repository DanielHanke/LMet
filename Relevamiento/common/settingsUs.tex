\usepackage{tabularx}
\usepackage[pdftex]{graphicx}
\usepackage[english]{babel}
\usepackage[utf8]{inputenc}
\usepackage{array}
%\usepackage{hyperref}

\usepackage{arimo}



%\usepackage[thinlines]{easytable}
%\usepackage[export]{adjustbox}
\usepackage{float}
\restylefloat{table}
\renewcommand{\arraystretch}{1.5}



\renewcommand*\familydefault{\sfdefault} %% Only if the base font of the document is to be sans serif
\usepackage[T1]{fontenc}
\usepackage[table]{xcolor}% http://ctan.org/pkg/xcolor

\newcommand{\todo}[1]{\tiny{\emph{to-do: #1}}\normalsize}

\usepackage{amssymb}
\usepackage{enumitem,xcolor}

\usepackage{tikz}
\usetikzlibrary{intersections}
\usepackage{epigraph}
\usepackage{lipsum}
\renewcommand\epigraphflush{flushleft}
\renewcommand\epigraphsize{\normalsize}
\setlength\epigraphwidth{0.7\textwidth}
\definecolor{titlepagecolor}{gray}{0.85}

\definecolor{blueTenaris}{RGB}{0,0,153}
\definecolor{greenTenaris}{RGB}{0,153,0}
\definecolor{fucsiaTenaris}{RGB}{204,0,102}

% las negritas en bluetenaris
\let\oldtextbf\textbf
\renewcommand{\textbf}[1]{\textcolor{blueTenaris}{\oldtextbf{#1}}}

% las italicas en bluetenaris
\let\oldemph\emph
\renewcommand{\emph}[1]{\textcolor{blueTenaris}{\oldemph{#1}}}

\usepackage{sectsty}
\sectionfont{\color{fucsiaTenaris}}  % sets colour of chapters
\subsectionfont{\color{blueTenaris}}  % sets colour of sections
\subsubsectionfont{\color{blueTenaris}}  % sets colour of sections

\DeclareFixedFont{\titlefont}{T1}{ppl}{b}{it}{0.5in}
\makeatletter                       
\def\printauthor{%                  
    {\large \@author}}              
\makeatother
%\author{%
%    Author 1 name \\
%    Department name \\
%    \texttt{email1@example.com}\vspace{20pt} \\
%    Author 2 name \\
%    Department name \\
%    \texttt{email2@example.com}
%    }
% This is the command I wish to optimize
\newcommand\titlepagedecoration[1]{%
\begin{tikzpicture}[remember picture,overlay,shorten >= -10pt]
\coordinate (tp1) at ([yshift=2cm]current page.west);
\coordinate (tp2) at ([yshift=2cm,xshift=9cm]current page.west);
\coordinate (tp3) at ([yshift=-15pt,xshift=7cm]current page.north);
\coordinate (tp4) at ([yshift=-15pt]current page.north west);
% Place text to het its coordinates
\node[right] (titletext) at ([xshift=1cm,yshift=-5cm]current page.north west) {\parbox{\textwidth}{\color{white}#1}};

\path[name path=p1] ([xshift=-5cm]tp2) -- ([xshift=-5cm]tp3);
\path[name path=p2] (tp2) -- (tp3);
\path[name path=p3] (tp1 |- titletext.south) -- (titletext.south -| tp3);

\path[name intersections={of=p1 and p3,name=first}];
\path[name intersections={of=p2 and p3,name=second}];

\filldraw[titlepagecolor!100!white] (first-1) -- (second-1) -- (tp3) -- ([xshift=-5cm]tp3) -- cycle;
\filldraw[titlepagecolor!80!white] (tp4) -- ([xshift=-5cm]tp3) -- (first-1) -- (tp1 |- titletext.south) -- cycle;
\filldraw[titlepagecolor!100!white] (tp1 |- titletext.south) -- (first-1) -- ([xshift=-5cm]tp2) -- (tp1) -- cycle;
\filldraw[titlepagecolor!80!white] (first-1) -- (second-1) -- (tp2) -- ([xshift=-5cm]tp2) -- cycle;

% X = -6 <-> 16

\filldraw[blueTenaris]  (-6,4.2) rectangle (2,4.3);
\filldraw[greenTenaris]  (2,4.2) rectangle (10,4.3);
\filldraw[fucsiaTenaris]  (10,4.2) rectangle (18,4.3);

%\filldraw[greenTenaris]  (-7,6) rectangle (2,3.5);
%\filldraw[fucsiaTenaris]  (-7,6) rectangle (2,3.5);

% Place text again, to have it on top
\node[right] (titletext) at ([xshift=1cm,yshift=-5cm]current page.north west) {\parbox{\textwidth}{\color{fucsiaTenaris}#1}};


\end{tikzpicture}%
}
