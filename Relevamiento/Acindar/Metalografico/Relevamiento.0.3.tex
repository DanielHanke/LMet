\documentclass{article}

\usepackage[automake]{glossaries}

\newcommand{\proyecto}[0]{LMet}
\newcommand{\actividad}[0]{Relevamiento ensayo metalográficos}
\newcommand{\docversion}[0]{0.3}
\newcommand{\fechado}[0]{Trend Ingeniería 2020}

\author{%
	\textbf{Autor}\\    
    Daniel E. Hanke \\
    Trend Ingenieria \\
    \texttt{dhanke@trendingenieria.com.ar}\vspace{20pt} \\
    }

\usepackage{tabularx}
\usepackage[pdftex]{graphicx}
\usepackage[spanish]{babel}
\usepackage[utf8]{inputenc}
\usepackage{array}
%\usepackage{hyperref}
\usepackage{listings}


%\usepackage{arimo}
%\usepackage{alegreya} %mas o menos%
\usepackage{arev} %mejor%
%\usepackage{bera}


%\usepackage[thinlines]{easytable}
%\usepackage[export]{adjustbox}
\usepackage{float}
\restylefloat{table}
\renewcommand{\arraystretch}{1.5}



\renewcommand*\familydefault{\sfdefault} %% Only if the base font of the document is to be sans serif
\usepackage[T1]{fontenc}
\usepackage[table]{xcolor}% http://ctan.org/pkg/xcolor

\newcommand{\todo}[1]{\tiny{\emph{to-do: #1}}\normalsize}

\usepackage{amssymb}
\usepackage{enumitem,xcolor}

\usepackage{tikz}
\usetikzlibrary{intersections}
\usepackage{epigraph}
\usepackage{lipsum}
\renewcommand\epigraphflush{flushleft}
\renewcommand\epigraphsize{\normalsize}
\setlength\epigraphwidth{0.7\textwidth}
\definecolor{titlepagecolor}{gray}{0.85}


% para trend!
\definecolor{blueTenaris}{RGB}{0,49,115}
\definecolor{greenTenaris}{RGB}{0,49,135}
\definecolor{fucsiaTenaris}{RGB}{0,49,115}



% las negritas en bluetenaris
\let\oldtextbf\textbf
\renewcommand{\textbf}[1]{\textcolor{blueTenaris}{\oldtextbf{#1}}}

% las italicas en bluetenaris
\let\oldemph\emph
\renewcommand{\emph}[1]{\textcolor{blueTenaris}{\oldemph{#1}}}

\usepackage{sectsty}
\sectionfont{\color{fucsiaTenaris}}  % sets colour of chapters
\subsectionfont{\color{blueTenaris}}  % sets colour of sections
\subsubsectionfont{\color{blueTenaris}}  % sets colour of sections

\DeclareFixedFont{\titlefont}{T1}{ppl}{b}{it}{0.5in}
\makeatletter                       
\def\printauthor{%                  
    {\large \@author}}              
\makeatother
%\author{%
%    Author 1 name \\
%    Department name \\
%    \texttt{email1@example.com}\vspace{20pt} \\
%    Author 2 name \\
%    Department name \\
%    \texttt{email2@example.com}
%    }
% This is the command I wish to optimize
\newcommand\titlepagedecoration[1]{%
\begin{tikzpicture}[remember picture,overlay,shorten >= -10pt]
\coordinate (tp1) at ([yshift=2cm]current page.west);
\coordinate (tp2) at ([yshift=2cm,xshift=9cm]current page.west);
\coordinate (tp3) at ([yshift=-15pt,xshift=7cm]current page.north);
\coordinate (tp4) at ([yshift=-15pt]current page.north west);
% Place text to het its coordinates
\node[right] (titletext) at ([xshift=1cm,yshift=-5cm]current page.north west) {\parbox{\textwidth}{\color{white}#1}};

\path[name path=p1] ([xshift=-5cm]tp2) -- ([xshift=-5cm]tp3);
\path[name path=p2] (tp2) -- (tp3);
\path[name path=p3] (tp1 |- titletext.south) -- (titletext.south -| tp3);

\path[name intersections={of=p1 and p3,name=first}];
\path[name intersections={of=p2 and p3,name=second}];

\filldraw[titlepagecolor!100!white] (first-1) -- (second-1) -- (tp3) -- ([xshift=-5cm]tp3) -- cycle;
\filldraw[titlepagecolor!80!white] (tp4) -- ([xshift=-5cm]tp3) -- (first-1) -- (tp1 |- titletext.south) -- cycle;
\filldraw[titlepagecolor!100!white] (tp1 |- titletext.south) -- (first-1) -- ([xshift=-5cm]tp2) -- (tp1) -- cycle;
\filldraw[titlepagecolor!80!white] (first-1) -- (second-1) -- (tp2) -- ([xshift=-5cm]tp2) -- cycle;

% X = -6 <-> 16

\filldraw[blueTenaris]  (-6,4.2) rectangle (2,4.3);
\filldraw[greenTenaris]  (2,4.2) rectangle (10,4.3);
\filldraw[fucsiaTenaris]  (10,4.2) rectangle (18,4.3);

%\filldraw[greenTenaris]  (-7,6) rectangle (2,3.5);
%\filldraw[fucsiaTenaris]  (-7,6) rectangle (2,3.5);

% Place text again, to have it on top
\node[right] (titletext) at ([xshift=1cm,yshift=-5cm]current page.north west) {\parbox{\textwidth}{\color{fucsiaTenaris}#1}};


\end{tikzpicture}%
}

\lstset{% general command to set parameter(s)
keywordstyle=\color{violet!55}\bfseries,
%keywordstyle=\color{blue}\bfseries\underbar,
% underlined bold black keywords
identifierstyle=\color{blue!60}, % nothing happens
commentstyle=\color{gray!70}, % white comments
stringstyle=\color{red!75}, % typewriter type for strings
numbers=left,
numberblanklines=false,
%frame=topline
backgroundcolor=\color{gray!10},
showstringspaces=false,
literate=*{0}{{\textcolor{violet}{0}}}{1}%
             {1}{{\textcolor{violet}{1}}}{1}%
             {2}{{\textcolor{violet}{2}}}{1}%
             {3}{{\textcolor{violet}{3}}}{1}%
             {4}{{\textcolor{violet}{4}}}{1}%
             {5}{{\textcolor{violet}{5}}}{1}%
             {6}{{\textcolor{violet}{6}}}{1}%
             {7}{{\textcolor{violet}{7}}}{1}%
             {8}{{\textcolor{violet}{8}}}{1}%
             {9}{{\textcolor{violet}{9}}}{1}%
             {.0}{{\textcolor{violet}{.0}}}{2}%
             {.1}{{\textcolor{violet}{.1}}}{2}%
             {.2}{{\textcolor{violet}{.2}}}{2}%
             {.3}{{\textcolor{violet}{.3}}}{2}%
             {.4}{{\textcolor{violet}{.4}}}{2}%
             {.5}{{\textcolor{violet}{.5}}}{2}%
             {.6}{{\textcolor{violet}{.6}}}{2}%
             {.7}{{\textcolor{violet}{.7}}}{2}%
             {.8}{{\textcolor{violet}{.8}}}{2}%
             {.9}{{\textcolor{violet}{.9}}}{2}%
             {=}{{\textcolor{violet}{=}}}{1}%
             {>}{{\textcolor{violet}{>}}}{1}%
             {<}{{\textcolor{violet}{<}}}{1}%
             {*}{{\textcolor{violet}{*}}}{1}%
             %{/}{{\textcolor{violet}{/}}}{1}% stuffs up comments
             {+}{{\textcolor{violet}{+}}}{1}%
             {-}{{\textcolor{violet}{-}}}{1}%
             {\%}{{\textcolor{violet}{\%}}}{1}%
             {:}{{\textcolor{violet}{:}}}{1}%
             {;}{{\textcolor{violet}{;}}}{1}%
             {,}{{\textcolor{violet}{,}}}{1}%
             {\&}{{\textcolor{violet}{\&}}}{1}%
             {(}{{\textcolor{violet}{(}}}{1}%
             {)}{{\textcolor{violet}{)}}}{1}%
             {\{}{{\textcolor{violet}{\{}}}{1}%
             {\}}{{\textcolor{violet}{\}}}}{1}%
             ,
} % no special string spaces

\renewcommand{\lstlistingname}{Ejemplo}% Listing -> Algorithm
\renewcommand{\lstlistlistingname}{Ejemplos de}

\makeglossaries

\newglossaryentry{TipoEntero}
{
name={\textbf{\textit{entero}}},
description={Valor numérico que puede contener solo números enteros, negativos, positivos o cero. Ej -8 0 75}
}

\newglossaryentry{TipoDecimal}
{
name={\textbf{\textit{decimal}}},
description={Valor numérico que puede contener solo números enteros de tipo real, es decir con un número arbitrario de decimales. Ej 0.871 -6,1 9 }
}

\newglossaryentry{TipoTexto}
{
name={\textbf{\textit{texto}}},
description={Valor alfanumérico de longitud arbitraria, puede contener números, letras y caracteres especiales tales como [  ¿ ! }
} 

\begin{document}

\input{../../Common/Titles/TrendTitle}

\tableofcontents
\newpage

\section{Versiones}

\begin{centering}
\begin{tabular}{|>{\centering}m{2cm}|m{3cm}|m{3cm}|m{6.5cm}|} \hline
\cellcolor{gray!25}Nro & \cellcolor{gray!25}Fecha& \cellcolor{gray!25}Autor & \cellcolor{gray!25}Descripción \\ \hline
0.1  & 20/11/2020 & DHanke	& Versión inicial \\ \hline
0.2  & 26/11/2020 & DHanke	& Agregado de glosario con tipos de datos para cada valor generado por ensayo \\ \hline
0.3  & 30/11/2020 & DHanke	& Agregado de entidades principales \\ \hline
\end{tabular}
\end{centering}

\newpage


%\includegraphics[width=1.0\textwidth]{./img/Diagram1.png}\\[0.1in]

\section{Introducción}

\subsection{Objetivo del documento}
\par El documento tiene por objeto relevar la solución de software del laboratorio metalográfico, pedido por la empresa Acindar, encargando este trabajo a Trend Ingeniería


\subsection{Audiencia}
\par Este documento está dirigido a aquellas personas que están involucradas en el sistema tanto en rol de usuario como en el desarrollo.


\section{Descripción de la nueva funcionalidad}

\subsection{Entidades}

{\setlength{\parindent}{0cm}\emph{Estudio}}

\par Un estudio es un pedido que hace un cliente al laboratorio, éste consta de múltiples ensayos que se realizaran en múltiples materiales y tiene como atributos:

\begin{itemize}
	\item Orden de fabricación
	\item Cliente
	\item Acero
	\item Otros...
\end{itemize}

{\setlength{\parindent}{0cm}\emph{Pastilla}}

\par Una pastilla es una porción de material a ser estudiada, y está relacionado con lo físico.

{\setlength{\parindent}{0cm}\emph{Muestra}}

\par Una muestra es una agrupación lógica dentro de una misma pastilla. Como forma de organización puede englobar uno o varios ensayos no relacionados.

{\setlength{\parindent}{0cm}\emph{Ensayo}}

\par Cada observación o medición que hace el profesional obteniendo valores. Estos a su vez determinarán la aceptación ( o no )


\subsection{Tipos de ensayos}

\par Para el laboratorio metalográfico, se definen los tipos de ensayos:
\begin{itemize}
	\item Decarburación  
	\item Dureza Brinell
	\item Microinclusiones
	\item Tamaño de grano
\end{itemize}

%----------------------------------------------------------------------------------------------------
\subsubsection{Ensayo de Decarburación  }

\par Valores generados por el ensayo
\begin{itemize}
	\item{Valor Mínimo} 
		\subitem Tipo de dato: \gls{TipoDecimal}
	\item{Valor Máximo} 
		\subitem Tipo de dato: \gls{TipoDecimal}
	\item{Tipo} 
		\subitem Tipo de dato: \gls{TipoEntero}
	\item{Observaciones}
		\subitem Tipo de dato: \gls{TipoTexto}
\end{itemize}
%----------------------------------------------------------------------------------------------------
\subsubsection{Ensayo de Dureza Brinell }

\par Valores generados por el ensayo
\begin{itemize}
	\item{Valor Mínimo} 
		\subitem Tipo de dato: \gls{TipoDecimal}
	\item{Valor Máximo} 
		\subitem Tipo de dato: \gls{TipoDecimal}
	\item{Observaciones}
		\subitem Tipo de dato: \gls{TipoTexto}
\end{itemize}
%----------------------------------------------------------------------------------------------------
\subsubsection{Ensayo de  Microinclusiones}
\par Valores generados por el ensayo
\begin{itemize}
	\item{Metodo A Fino}
		\subitem Tipo de dato: \gls{TipoDecimal}
	\item{Metodo A Grande}
		\subitem Tipo de dato: \gls{TipoDecimal}
	\item{Metodo B Fino}
		\subitem Tipo de dato: \gls{TipoDecimal}
	\item{Metodo B Grande}
		\subitem Tipo de dato: \gls{TipoDecimal}
	\item{Metodo C Fino}
		\subitem Tipo de dato: \gls{TipoDecimal}
	\item{Metodo C Grande}
		\subitem Tipo de dato: \gls{TipoDecimal}
	\item{Metodo D Fino}
		\subitem Tipo de dato: \gls{TipoDecimal}
	\item{Metodo D Grande}
		\subitem Tipo de dato: \gls{TipoDecimal}
\end{itemize}

%------------------------------------------------------------------------------------------------
\subsubsection{Ensayo de Tamaño de grano }
\par Valores generados por el ensayo
\begin{itemize}
	\item{Tamaño}
		\subitem Tipo de dato: \gls{TipoEntero}
\end{itemize}
%------------------------------------------------------------------------------------------------

\newpage
\printglossaries


%Objetos de la base de datos
%\begin{itemize}
%\item Usuarios
%\item Permisos ( de carga de info, de lectura )
%\item Estudio Conjunto de pastillas
%\item Pastilla Conjunto de muestras
%\item Muestra (Conjunto de ensayos? Uno a uno?)
%\item Tipos de muestras (Definen los ensayos a realizar? )
%\item Orden de fabricación
%\item Ensayos
%\item Resultados
%\subitem Cuantitativos Rutinas de aptitud - reensayos ( que significa esto? ) explicarse
%\subitem Cualitativos - Reportes con datos + imagenes, necesitan ser estandarizados
%\end{itemize}
%
%\subsubsection{funcionalidades}
%
%\par Ensayar ( Cargar info de resultados ) Mostrar info de recetas setup de los ensayos
%\par Imprimir codigo QR acompañando la muestra
%\par Generar estadísticas ( buenos / malos ?) cantidades graficos de torta, otros?
%\par Generar las hojas de ruta ( que ensayos debe hacerse para cada muestra ) Cual es el criterio? Son todas iguales? hay muestras que hay que hacer A, b, otra cosa?
%\par Consultar Historico
%\par Consultar estadisticas



%
%\par A continuación la tabla que envió IT: \\
%
%\begin{centering}
%\begin{tabular}{|>{\centering}m{3cm}|m{3cm}|m{6.5cm}|} \hline
%\cellcolor{gray!25}ID & \cellcolor{gray!25} D-REDUCI-OPR & \cellcolor{gray!25}D-COMPLE-OPR \\ \hline
%209	& COLCUP	& COLOCAR CUPLAS (PEC) \\ \hline
%321	& GP CUP	& GLASS PEENING CUPLAS \\ \hline
%428	& DOPELP	& TRATAM. DOPELESS PIN \\ \hline
%457	& DOPFRE	& TRAT.ROSC.API DOPFRE \\ \hline
%470	& LOTCUP	& CTRL LOTE CUPLA \\ \hline
%471	& COLVAL	& CTRL COL CUPLA VALID \\ \hline
%472	& COLCIC	& CTRL COL/CICLO CUPLA \\ \hline
%473	& APRCUP	& CTRL APR LOTE/COLCUP \\ \hline
%474	& RFIDBX	& CTRL RFID BOX \\ \hline
%475	& RFIDPN	& CTRL RFID PIN \\ \hline
%476	& CTRLDM	& CTRL DATA MATRIX \\ \hline
%477	& CTRURC	& CTRL NUMERACION URC \\ \hline
%478	& HDRMEM	& FALTA EMI HDRM \\ \hline
%479	& HDRMUT	& FALTA UT HDRM \\ \hline
%480	& HDRMWM	& FALTA WMPI HDRM \\ \hline
%481	& HDRMPH	& FALTA PH HDRM \\ \hline
%\end{tabular}
%\end{centering}
%
%\subsubsection{Incidencias de Nivel 2}
%\par Estos son los chequeos expandidos de nivel 2
%
%\begin{centering}
%\begin{tabular}{|>{\centering}m{3cm}|m{3cm}|m{6.5cm}|} \hline
%\cellcolor{gray!25}IDChequeo & \cellcolor{gray!25} IDSubChequeo & \cellcolor{gray!25} Descripcion \\ \hline
%1  & 1 & SEMI \\ \hline
%10 & 1 & ELENCO\_ESTARCIDO \\ \hline
%11 & 1 & CAR\_LOAD \\ \hline
%12 & 1 & TRACEABILITY \\ \hline
%13 & 1 & DPLS \\ \hline
%13 & 2 & DPLS\_API \\ \hline
%2  & 1 & RFID\_PIN \\ \hline
%2  & 2 & RFID\_BOX \\ \hline
%2  & 3 & DM\_RADIAL \\ \hline
%2  & 4 & TRAC\_ETIQUETA \\ \hline
%3  & 1 & HDRM \\ \hline
%6  & 1 & CUPLA\_VIR \\ \hline
%7  & 1 & CICLO\_COLADA \\ \hline
%7  & 2 & CUPLA\_LOTE \\ \hline
%7  & 3 & CUPLA\_UNION \\ \hline
%8  & 1 & RETE \\ \hline
%\end{tabular}
%\end{centering}
%
%
%
%
%\section{Desarrollo}
%\subsection{Chequeos actuales nivel 2}
%
%\subsection{Incidencias IT}
%
%\subsection{Configuración IT - Nivel 2}
%\par Se describe aquí la correspondencia entre IT y Nivel 2, generando chequeos nuevos, incluso, si aquellos chequeos que no entran en ninguno antes desarrollado.
%
%\begin{tabular}{ccc}
%\hline 
%Chequeo IT 				& Chequeo N2 			& Descripción \\ \hline
%COLOCAR CUPLAS (PEC) 	& & \\
%GLASS PEENING CUPLAS 	& & \\
%TRATAM. DOPELESS PIN 	& & \\
%TRAT.ROSC.API DOPFRE 	& & \\
%CTRL LOTE CUPLA 		& & \\
%CTRL COL CUPLA VALID 	& & \\
%CTRL COL/CICLO CUPLA 	& & \\
%CTRL APR LOTE/COLCUP 	& & \\
%CTRL RFID BOX 			& RFID\_BOX 			& \\
%CTRL RFID PIN 			& RFID\_PIN 			& \\
%CTRL DATA MATRIX 		& DM\_RADIAL 			& \\
%CTRL NUMERACION URC 	& TRAC\_ETIQUETA 		& \\
%FALTA EMI HDRM 			& & \\
%FALTA UT HDRM 			& & \\
%FALTA WMPI HDRM 		& & \\
%FALTA PH HDRM 			& & \\
%\hline 
%\end{tabular} 
%
%
%\subsection{Interfaz gráfica}
%\subsubsection{Cambios afectados}
%
%\section{Apéndices}
%
%\subsection{Documentos anexos}
%
%\subsection{Dudas y consultas}
%
%\begin{itemize}[label=\textcolor{greenTenaris}{$\checkmark$}]
%\item Como asignar las tablas IT versus nivel 2
%\end{itemize}

\end{document}